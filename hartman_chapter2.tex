%
%Global Physical Climatology 
%Dennis L. Hartman
%                2016年卒業研究資料 
%                 塩尻千里
%
%
% 2016/10/06 〜 作成
%
% 日付を変更すること
%
%%%%%%%%%%%%%%%%%%%%%%%%%%%%%%%%%%%%%%%%%%%%%%%%%%%%%%%%
%%%%%%%%             Style  Setting             %%%%%%%%
% フォント: 12point (最大), 片面印刷
\documentclass[a4j,12pt,openbib,oneside,dvipdfmx]{jbook}

%%%%%%%%%%%%%%%%%%%%%%%%%%%%%%%%%%%%%%%%%%%%%%%%%%%%%%%%
%%%%%%%%             Package Include            %%%%%%%%
\usepackage{Dennou6}		% 電脳スタイル ver 6
\usepackage{ascmac}
\usepackage{tabularx}
\usepackage{graphicx}
\usepackage{amssymb}
\usepackage{amsmath}
\usepackage{bm}

%%%%%%%%%%%%%%%%%%%%%%%%%%%%%%%%%%%%%%%%%%%%%%%%%%%%%%%%
%%%%%%%%            PageStyle Setting           %%%%%%%%
\pagestyle{DAmyheadings}


%%%%%%%%%%%%%%%%%%%%%%%%%%%%%%%%%%%%%%%%%%%%%%%%%%%%%%%%
%%%%%%%%        Title and Auther Setting        %%%%%%%%
%%
%%  [ ] はヘッダに書き出される.
%%  { } は表題 (\maketitle) に書き出される.

\Dtitle{Hartman(1994)}   % 変更不可
\Dauthor{塩尻千里}       % ゼミ担当者の名前
%\Ddate{2016/10/19}      % ゼミの日時 (毎回変更すること)
\Dfile{hartman\_chapter2.tex}  %file名




%%%%%%%%%%%%%%%%%%%%%%%%%%%%%%%%%%%%%%%%%%%%%%%%%%%%%%%%
%%%%%%%%        Counter Output Format           %%%%%%%%
\def\thechapter{\arabic{chapter}}
\def\thesection{\arabic{chapter}.\arabic{section}}
\def\thesubsection{\arabic{chapter}.\arabic{section}.\arabic{subsection}}
\def\theequation{\arabic{chapter}.\arabic{equation}}
\def\thepage{\arabic{page}}
\def\thefigure{\arabic{chapter}.\arabic{figure}}
\def\thetable{\arabic{chapter}.\arabic{section}.\arabic{table}}
\def\thefootnote{*\arabic{footnote}}

%%%%%%%%%%%%%%%%%%%%%%%%%%%%%%%%%%%%%%%%%%%%%%%%%%%%%%%%
%%%%%%%%   Set Counter (chapter, section etc. ) %%%%%%%%
\setcounter{chapter}{1}    % 章番号
\setcounter{section}{0}    % 節番号
\setcounter{subsection}{0}    % 節番号
\setcounter{equation}{0}   % 式番号
\setcounter{page}{1}     % 必ず開始ページは明記する
\setcounter{figure}{0}     % 図番号
\setcounter{table}{0}      % 表番号
\setcounter{footnote}{0}

%%%%%%%%%%%%%%%%%%%%%%%%%%%%%%%%%%%%%%%%%%%%%%%%%%%%%%%%
%%%%%%%%        Dennou-Style Definition         %%%%%%%%

%% 改段落時の空行設定
\Dparskip      % 改段落時に一行空行を入れる
%\Dnoparskip    % 改段落時に一行空行を入れない

%% 改段落時のインデント設定
\Dparindent    % 改段落時にインデントする
%\Dnoparindent  % 改段落時にインデントしない

%%%%%%%%%%%%%%%%%%%%%%%%%%%%%%%%%%%%%%%%%%%%%%%%%%%%%%%%
%%%%%%%%             Text Start                 %%%%%%%%
\begin{document}
\markright{2.1 熱とエネルギー}

\chapter{全体のエネルギーバランス}	
\section{熱とエネルギー}
\par
気候変動の鍵である温度は, 分子運動に含まれるエネルギーの単位である. どのように温度が維持されるかを理解するためには, 熱力学第一法則で正式に述べられているエネルギーバランスを考えなければならない. 
基本的な地球の全体のエネルギーバランスは, 太陽からくるエネルギーと, 地球の放射によって宇宙空間に戻されるエネルギー間のつり合いである. 地球内部におけるエネルギーの生成は, そのエネルギー収支においては無視できるほどの影響である. 
太陽放射の吸収はほとんど地表面で行われるが, 一方で, 宇宙空間への放射のほとんどは大気から生じる. 地球大気は効率的に赤外線放射を吸収, 射出するので, 地表面は大気が存在しなかった場合に比べてより温暖である. 
一年以上にわたって平均すると, 太陽エネルギーは, 極付近よりも赤道付近でより多く吸収されている. 大気と海洋は, 表面温度に対する熱の勾配の効果を減らすように極へエネルギーを輸送する. 地球の進化や気候の特徴の大半は, 太陽系内における地球の位置によって決定されてきた.

\newpage
\section{太陽系}
\markright{2.2 太陽系}
\par
地球上で生命を維持するためのエネルギー源は太陽からくる. 地球は太陽からの距離を比較的一定に保ちながら一年をかけて太陽の周囲を軌道を描いて回る. そのため私たちの太陽は安定で快適な熱と光の資源を提供する. 私たちの太陽は, 私たちの銀河である天の川において, 約$10^{11}$個ある星のうちの一つである. それは一つの星であるが, 一方で, 目に見える星の$2/3$は多重星の系にある.
\par
光度とは, 太陽によってエネルギーが放射されたところにおける合計の比率である. 
太陽に比べて$10^{-4}$倍の光度がある恒星や, $10^5$倍の輝度をもつ恒星があることを私たちは知っている. それらの温度は 2000 ~ 30000 K にわたって分布している一方で, 太陽の光球の温度はおよそ 6000 K である. 
光球は, 太陽のエネルギー放射の大部分を宇宙空間に射出する領域である. 恒星の半径は, 0.1 から 200 太陽半径に及ぶ. エネルギーは核融合によって太陽の核で生成され, そこでより軽い元素が重い元素にされ, その過程でエネルギーの放出が伴う.
太陽のようなやや小さい星の場合, 主な順序において予想される寿命はおよそ 110 億年であり, およそ半分が過ぎてしまった. 従って太陽はさびしく, 中年で, 中くらいの輝きの星である. 
星の進化の理論は, 太陽の光度は地球の一生, およそ 50 億年の間, 約 30$\%$で増加してきたと予測している. 
\par
太陽系は 9 つの惑星を含む. それらは地球型もしくは内側の惑星と, ヨブもしくは外側の惑星(図2.2)とに分類されうる. 地球型の惑星には, 水星, 金星, 火星, 地球が含まれる. ヨブの惑星には木星, 土星, 天王星, 海王星が含まれる. 冥王星はどちらのカテゴリーのもきちんと分類されない. \\
\subsection{惑星の運動}
\par
惑星は太陽の周りを楕円軌道を描いて回る. そしてそれには三つの特徴がある: 惑星太陽間の距離の平均, 離心率, 軌道面の向きである. 太陽からの距離の平均は, 惑星に到達する太陽のエネルギーフラックス密度(単位時間あたり単位面積あたりに届くエネルギー)の量を決める. 太陽からの平均距離はまた, 惑星の一年つまり惑星が一回軌道を回り切るのにかかる時間を決める. 惑星の一年は太陽からの距離が大きくなるにつれて増える.
\par
軌道の離心率とは, 完全な円軌道からどれだけ軌道がそれているかの程度のことである. 惑星が惑星の一年間にその軌道を動くのに伴う, 惑星に到達する太陽のエネルギーフラックス密度の変化量を決める. 
もし離心率がゼロ(円軌道)でないならば, 太陽から惑星までの距離は一年を通じて変化する. 軌道面の向きは気候に対してそこまで支配的な影響がない. 冥王星を除いた多くの惑星は, 程度の差はあれど同じ軌道面にいる.
\par
軌道の要素に加えて, 惑星の自転の要素と軌道との関係はとても重要である. 自転速度は, ある地点における日射の時間的な振る舞い(日変化)を決める. 
そしてまた自転速度は, 太陽の加熱に対する大気と海洋の応答を決める上で重要であり, それによって風と発達する流れのパターンを決める.
\par
傾斜もしくは傾きは, 自転軸と, 軌道面の法線方向間の角度である. それはとりわけ高緯度において, 日射の季節変化に影響を与える. それはまた極域に届く日射の年間平均に強く影響する. 現在地球の自転軸の傾きはおよそ$23.45^\circ$である.
\par
近日点の長さは, 軌道上における惑星の位置に関連した季節の段階を示す. たとえば現在は, 南半球が夏の間, およそ 1 月 5 日に地球は太陽の最も近くを通る(近日点). 結果として南半球は北半球に比べてより多く, 大気上層の日射を受け取る.
\par
気象におけるこれらの軌道の要素の効果は, 11 章でより詳しく議論されるだろう. そこでは気候変動に関する軌道の要素の理論が記述されている. ここでは太陽からの距離と, 傾斜もしくは偏角だけを考える.

\newpage
\section{地球のエネルギーバランス}
\markright{2.3 地球のエネルギーバランス}
\subsection{熱力学第一法則}
\par
熱力学第一法則はエネルギーが保存することを述べている. 閉じた系における第一法則は, 「ある系に加えられた熱は内部エネルギーの変化マイナスなされた仕事に等しい」と記述されるかもしれない. この法則は, 記号では
\begin{equation}
  dQ=dU-dW
\end{equation}
と記述されうる.
\par
ここで$dQ$は加えられた熱量, $dU$は系の内部エネルギーの変化, $dW$は系になされた仕事である.
\par
熱は三つの方法で系に輸送され, 輸出されうる.
\begin{enumerate}
  \item{\textgt{放射}: 質量が変化せず, 媒質は必要とされない. 純粋なエネルギーは光速で動く.}
  \item{\textgt{伝導}: 質量は変化しないが, 原子もしくは分子同士の衝突によって熱を輸送するため. 媒質が必要とされる.}
  \item{\textgt{対流}: 質量が変化する. 正味の質量の運動が起こる. しかしより一般には, 異なるエネルギー量をもつパーセルが場所を変えるので, 正味の質量の運動は起こらず, エネルギーが変化する(図2.1).}
\end{enumerate}
\par
太陽から地球へのエネルギーの伝達はほとんどすべて放射である. 質量フラックスは太陽風の粒子と関連するものもあるが, エネルギー量は地球の表面温度に対して重要な効果をもつには小さすぎる. 
さらに, 地球環境において地球によってなされた仕事もまた, 無視できるほどである. 地球のエネルギーバランスを近似的に計算するためには, 放射エネルギーのみを考える必要がある. 
地球と太陽の間のエネルギーの流れに影響を与えうるような宇宙空間の物質の量は小さい. 従って, 太陽の光球と地球大気の上端との間の宇宙空間を真空と考える. 
真空では放射だけがエネルギーを輸送しうる. 

\subsection{エネルギーフラックス, フラックス密度, 太陽定数}
\par
太陽はほぼ一定でエネルギーフラックスを放出しており, それを私たちは光度とよび, $L_0=3.9\times10^{26}$ Wである.
エネルギーフラックスを光球の面積で割ることで, 光球におけるフラックス密度の平均を計算できる. 
\begin{align}
  {\rm{フラックス密度(光球)}}&=\frac{\rm{エネルギーフラックス}}{\rm{光球の表面積}}=\frac{L_0}{4\pi{r^2_{\rm{photo}}}} \nonumber \\
  &=\frac{3.9\times10^{26}\,\rm{W}}{4\pi\left[6.96\times10^8\,\rm{m}\right]^2}=6.4\times10^7\,\rm{Wm^{-2}}
\end{align}
\par
宇宙空間はだいたい真空であり, エネルギーは保存されるので, 太陽が中心にあるすべての球面を通じて外向きに通過するエネルギーの量は, 光度もしくは太陽からのエネルギーフラックスの合計に等しい. フラックス密度が球上で一様だと仮定し, 太陽からある距離$d$におけるフラックス密度を$S_d$と書くと, エネルギー保存より,
\begin{equation}
  {\rm{エネルギーフラックス}}={L_0}={S_d}4\pi{d^2}
\end{equation}
である.
\par
これより, フラックス密度は太陽までの距離の平方根の逆数に比例することが分かる. いま太陽定数をある特定の距離における太陽放射のエネルギーフラックス密度と定義する.
\begin{equation}
  {\rm{太陽定数}}={\rm{距離 d におけるフラックス密度}}\, d=S_d=\frac{L_0}{4\pi{d^2}}
\end{equation}
太陽定数は, ある固定された半径かつ太陽を中心とする球面でのみ一定である. 太陽から地球までの平均の距離($1.5\times10^{11}\,\rm{m}$)においては, 太陽定数は$S_0=1367\,\rm{Wm}^{-2}$である.

\subsection{空洞の放射}
\par
熱力学的に平衡な状態にある, ある閉じた空洞内の放射領域は, 空洞の壁の温度それぞれに相関のある値をもち, 空洞がつくられている物質には関係しない. 
この空洞の放射強度, それは壁の温度それぞれに相関があるのだが, それはまた黒体放射と呼ばれる. なぜなら, 表面からの放射に単位射出量が一致するからである. 
完全な黒体は容易に見つからないかもしれないが, 平衡状態にある空洞内部の放射は常に黒体放射と等しいだろう. 
黒体放射の温度の依存性はステファンボルツマンの法則に従う.
\begin{equation}
  {\rm{E_{BB}}}=\sigma{T^4}\,\,;\,\,\sigma=5.67\times10^{-8}\,\rm{Wm}^{-2}\rm{K}^{-4}
\end{equation}

\subsubsection{例: 太陽の放射温度}
光球における太陽のフラックス密度はおよそ$6.4\times10^7\,\rm{Wm}^{-2}$と私たちは容易に計算した. 私たちはこれをステファンボルツマンの式と等しくすることができ, 光球についての有効放射温度を導くことができる. 

\subsection{射出率}
平衡状態において, 温度$T$における空洞内部の放射強度は$E_{BB}=\sigma{T^4}$である. 
私たちは射出量を, 黒体放射に対する同じ温度における気体の密度もしくは体積の実際の射出の割合として$\varepsilon$を定義する.
\begin{equation}
  E_R=\varepsilon\sigma{T^4} \,\,\to\,\, \varepsilon=\frac{E_R}{\sigma{T^4}}
\end{equation}
\newpage
\markright{2.4 惑星の放射温度}
\section{惑星の放射温度}
惑星の放射温度は, エネルギーバランスを満たすために, 黒体は放射する必要があり, それに伴う黒体の温度である. 基本的な概念は, 惑星によって吸収される太陽エネルギーと黒体によって射出されるエネルギーが等しいということである. これは惑星の放射温度を定義する.
\begin{equation}
  \rm{吸収される太陽放射}=\rm{射出される惑星の放射} \nonumber
\end{equation}
\par
吸収される太陽放射を計算するために, 太陽定数から始めることにする. それは太陽惑星間の平均距離に到達する太陽放射のエネルギーフラックス密度を示す. 
フラックス密度は, 放射の方向と垂直を成す平らな地表面と関連して定義される. 
太陽放射は基本的に, 太陽系においては, 惑星の本体に対し, 平行かつ一定の光線である.
なぜならすべての惑星が太陽からの距離と比べて直径が小さいからである. 
惑星に到達するエネルギーの量は, 平行なエネルギーフラックスの光線を惑星がはきだす面積に太陽定数をかけたもの等しい. これをシャドウエリアとよぶ(図2.2). 地球大気はとても薄いので, シャドウエリアに対する影響は無視することができ, その面積を計算するために, 惑星の半径$r_p$を用いることができる. 
\par
わたしたちはまた, 惑星に届く太陽エネルギーのすべてが吸収されるわけではないことに注意しなければならない. 一部は吸収されることなく, 宇宙空間に射出されるものもあるので, それは惑星のエネルギーバランスには入らない. この惑星の反射率をアルベドとよび, 記号$\alpha_p$で表す. 従って,
\begin{equation}
  {\rm{吸収された太陽放射}}=S_0(1-\alpha_p)\pi{r_p}^2.
\end{equation}
\par
大気上端において, 全球で平均された日射はおよそ$342 \rm{W}\rm{m}^{-2}$である. 地球の惑星アルベドは$30 \%$なので, $70 \%$だけが気候システムによって吸収されており, およそ$240 \rm{W}\rm{m}^{-2}$である. 
これだけのエネルギーが地球放射によって宇宙空間に戻されなければならない. 地球放射が黒体の放射のようであると仮定することにする. 放射がおこる面積は, 円の面積ではなく球の表面積である. 従って, 地球の放射フラックスは,
\begin{equation}
  \rm{射出される地球放射}=\sigma{T_e}^{4}4\pi{r_p}^2
\end{equation}
と書かれる.
\par
吸収される太陽フラックスと射出される地球放射フラックスが等しいなら, 惑星のエネルギーバランスを得るだろう. そしてそれは放射温度を定義するだろう. 
\begin{equation}
  \frac{S_0}{4}(1-\alpha_p)=\sigma{T_e}^4
\end{equation}
もしくは
\begin{equation}
  T_e=\left(\frac{S_0(1-\alpha_p)}{4\sigma}\right)^{\frac{1}{4}} \label{te}
\end{equation}
と書ける.
\par
太陽定数を割っている$4$の因子は, シャドウエリアに対する球の全体の表面積の割合である. 放射温度は惑星の表面もしくは大気の実際の温度ではないかもしれない; それは, 吸収する太陽エネルギーとつりあうために惑星が必要とする, 黒体の放射温度にすぎない. 
\subsubsection{地球の放射温度}
地球のアルベドはおよそ$0.3$である. \eqref{te}より, 地球の放射温度は$255 \rm{K}$である. $255 \rm{K}$という放射温度は, 観測された全球の平均温度, $288 \rm{K}$よりかなり低い. この差を理解するためには温室効果を考える必要がある.

\newpage
\markright{2.5 温室効果}
\section{温室効果}
放射温度を定義するのに用いられる, とても単純なエネルギーバランスモデルの記述で説明されるものの一つに温室効果がある. 地球放射に関しては黒体であることを仮定しているが, 太陽放射に対しては透過する大気で, 全球のエネルギーバランスは考えられている(図2.3).
\par
太陽放射はほとんど可視光であり, 近赤外線である. そして地球は主に赤外放射で熱を射出する. 大気は, 太陽と地球の放射それぞれに影響を与えうる. 
このモデルにおける大気上端でのエネルギーバランスは, (2.9)の放射温度を定義する基本的なエネルギーバランスモデルと同じである. 大気の層は, その下で地表によって射出されるエネルギーのすべてを吸収し, 黒体のように射出するので, 宇宙空間に射出される放射は, このモデルにおいては大気から射出されるものだけである. 
従って, 大気上端でのエネルギーバランスは
\begin{equation}
 \frac{S_0}{4}(1-\alpha_p)=\sigma{T_A}^4=\sigma{T_e}^4 
\end{equation}
である.
\par
従って, 平衡状態にある大気の温度はエネルギーバランスを満たすために放射温度でなければならない. 
しかし, 大気と地表のエネルギーバランスを考えることでみられるように, 表面温度はより温かい. 大気のエネルギーバランスを
\begin{equation}
  \sigma{T_S}^4=2\sigma{T_A}^4\,\,\to\,\,\sigma{T_S}^4=2\sigma{T_e}^4
\end{equation}
と与えると, 地表のエネルギーバランスは矛盾しない:
\begin{equation}
  \frac{S_0}{4}(1-\alpha_p)+\sigma{T_A}^4=2\sigma{T_S}^4\,\,\to\,\,\sigma{T_S}^4=2\sigma{T_e}^4 \label{et}
\end{equation}
\par
図2.3 にある図と\eqref{et}で与えられる地表のエネルギーバランスから, 大気は地表への太陽エネルギーの流れを妨げない, そして大気は長波放射の下向きの射出によって地表面の太陽による加熱つまりこの場合太陽を増加させるので, 地表面の温度は増加していく. 大気は相対的に太陽放射を輸送し, その上とても効果的に地球放射を吸収, 射出するので, 大気の温室効果は地表面を温める.

\newpage
\markright{2.6 全球の放射フラックスのエネルギーバランス}
\section{全球の放射フラックスのエネルギーバランス}
大気におけるエネルギーの鉛直方向のフラックスは, 多くの重要な気候のプロセスのうちの一つである. 地表, 大気, 宇宙空間の間での放射と放射ではないフラックスは, 気候の決定の鍵である. 太陽放射の大気の通過しやすさと, 地球放射の透過しづらさは, 温室効果の強さを決める大気を通じて伝えられる.
\par
地表, 対流圏, 成層圏のエネルギーバランスへの放射のプロセスの寄与は, 概略的に図2.4 に示される. 大気上端において, 全球で平均した利用できる太陽放射の割合でその値は決まり, およそ$342 \rm{W}\rm{m}^{-2}$である. 
惑星は太陽放射のおよそ$70\%$を吸収し, $30\%$を反射する. 大気上端で利用できる日射の$50\%$は完全に地表に到達し, そこで吸収される. 成層圏で吸収される$3\%$は主にオゾンと酸素分子によるものである一方で, 二酸化炭素と水蒸気はおよそ$0.5\%$に寄与する. 
対流圏で吸収される太陽放射の$17\%$は, ほとんど水蒸気($13\%$)と雲($3\%$)によるものである一方で, 二酸化炭素, オゾン, 酸素は合わせて, 残りの$1\%$に寄与する. 
\par
図 2.4 の顕著な特徴は, 長波放射による, 地表と大気間の内部の交換が全体で最も規模が大きく, また大気上端における日射よりも大きいことである. 
このことは地球大気における温室効果の重要性を示している. 対流圏において長波放射の吸収射出に主要な働きをするのは, 水蒸気, 雲, オゾン, 亜酸化窒素, メタン, それと他のいくつかの少数派の物質である. 水蒸気と雲は, 現在の温室効果のおよそ$80\%$を担う. 
大気の温室効果の良い指標は, 図の長波放射の部分の左の柱であり, そしてそれは地表から射出された$110$ユニットのうち, たったの$10$ユニットだけが, 大気に吸収射出されることなく, 宇宙空間に通過することを示している. 
対流圏は, 放射と放射でない資源から$149$ユニットを受け取る: 長波放射のエネルギー吸収が$103$ユニット, 潜熱の開放が$24$ユニット, 太陽放射の吸収が$17$ユニット, 地表から輸送される顕熱が$5$ユニットである. 
対流圏は地表に向かって$89$ユニット再び射出し, 成層圏と宇宙空間に上向きに$60$ユニット射出する.
\par
大気からの地球放射のとても強い下向きの射出は, 地表面温度の比較的小さな日変化を維持するために不可欠である. 下向きの長波放射が, 地表を加熱する太陽放射よりも大きくなかったとしたら, 陸の地表面温度は夜に急速に冷え, 日中は非常に急速に温められただろう. 温室効果は地表の温度を比較的温かく保つだけでなく, また陸での地表面温度の日変化の振れ幅を抑制する.

\newpage
\markright{2.7 日射の配分}
\section{日射の配分}
温度の季節的, 緯度的な変化は主に日射と平均の太陽天頂角の変化によって起こる. 大気上端における太陽放射の量は, 緯度, 季節, 一日の時間に依存する. 吸収されずに宇宙空間に反射する太陽の放射エネルギーの量は, 太陽天頂角と, その土地の地表と大気の性質によって決まる. 気候は 24 時間, 一つの季節, 一年にわたって平均された日射と天頂角に依存する. この節では, 日射と太陽天頂角で決まる幾何学的な要因が記述されるだろう.
\par
地球の平均位置における単位面積あたりの平均太陽フラックス, $1367\,\rm{Wm}^{-2}$は太陽光線に対して垂直である地表面で測られる. 
地球は近似的には球なので惑星の表面の多くは, 太陽光線に対して傾いた角度になる傾向にある. 
そのような場合, 単位表面積あたりのフラックスは太陽のフラックス密度よりも小さいので, 太陽フラックスの密度は垂直なときの面積よりも広い面積に広がる. 
地球の表面上の, ある垂線と, 地球の表面上のある点と太陽とを結ぶ線との間の角度として太陽天頂角$\theta_s$を定義する. 
図 2.5 は, 地表の表面積に対するシャドウエリアの比が太陽天頂角のコサインに等しいことを示している. 
単位表面積あたりの太陽フラックスは,
\begin{equation}
  Q=S_0\left(\frac{\overline{d}}{d}\right)^2\cos{\theta_s}
\end{equation}
と書ける. ここで$\overline{d}$は太陽フラックス密度$S_0$が測られたところの太陽からの平均距離, $d$は太陽からの実際の距離である.
\par
太陽天頂角は緯度, 季節, 一日の時間に依存する. 季節は太陽赤緯すなわち正午に太陽の直下にある地球表面上の地点の緯度によって表現されうる. 
太陽赤緯($\delta$)は現在, 北半球の夏至(6/21)における$+23.45^\circ$と北半球の冬至(12/21)における$-23.45^\circ$の間で異なる.
時角$h$は正午における太陽位置から測った太陽の方位の経度として定義される.
これらの定義を行うなら, 太陽天頂角のコサインは, 球面三角法より, 緯度, 季節, 一日の時間から得られる(付録 A を見よ).
\begin{equation}
  \cos{\theta_s}=\sin\phi\sin\delta+\cos\phi\cos\delta\cos{h} \label{cos}
\end{equation}
太陽天頂角が負の場合, 太陽は水平線より下にあり, 地球表面は暗闇である.
日の出と日没は太陽天頂角が$90^\circ$のときに起こり, この場合\eqref{cos}は
\begin{equation}
  \cos{h_0}=-\tan\phi\tan\delta
\end{equation}
となる. ここで$h_0$は日の出と日没時の時角である.
\par
極付近では, 特別な状況が起こる. 緯度と太陽赤緯が同符号のとき(夏), $90-\delta$度の極よりの緯度は一定で輝く. 
極においては, 太陽は水平線上で一定の角度$\delta$の範囲のまわりを動く.
冬半球においては, $\phi$と$\delta$は異符号であり, $90-|\delta|$度の極よりの緯度は極の暗闇にある.
極では, 6 ヶ月の暗闇と日に照らされる 6 ヶ月とが交互に起こる. 赤道では一年を通じて昼と夜の長さは両方 12 時間である.
\par
大気上端のある緯度の表面における一日の日射の平均は, (2.15)を(2.14)に代入し, 日の出と日没間で結果を積分し, 24 時間で割ることで得られる. 
その結果が
\begin{equation}
  \overline{Q}^{day}=\frac{S_0}{\pi}\left(\frac{\overline{d}}{d}\right)^2\left[h_0\sin\phi\sin\delta+\cos\phi\cos\delta\cos{h_0}\right]
\end{equation}
である. ここで日の出と日没の時角$h_0$はラジアンで与えられなければならない. 一日の平均日射量は, 緯度と季節の関数として図 2.6 にプロットされている.
地球の軌道は厳密には円ではない. 正確には地球は, 北半球が夏の間に比べて, 南半球が夏の間の方が太陽に多少近い. 
結果として南半球における日射の最大値は, 北半球に比べておよそ$6.9\%$高い.
夏至のとき, 高緯度における日射が赤道付近よりも実際は多いことに注意せよ.
このことは夏の間で最も長い一日であることの結果であり, 高緯度において天頂角が比較的大きいことの結果ではない.
\par
一日の日射が一年全体にわたって平均されると, 図 2.7 のような分布が得られる.
極における大気上端での日射の年平均は, 赤道における値の半分以下であり, 赤道で最大値に達する. 赤道での日射の年平均と至での日射を比べると, 赤道での日射は, 昼夜の時間が等しいときの最大値と至での最小値によって年二回の変動を切り抜けていることが分かる.
\par
地球の局所的なアルベドは太陽天頂角に依存するので, 天頂角は単位表面積あたりの利用できるエネルギーとアルベドの両方の決定に入り込む. 
従って, 緯度と季節の関数として日中の時間で平均した太陽天頂角を考えることは興味深い.
一日の天頂角の平均を計算すると, 時間よりもむしろ日射と関連して, 平均に重しをつけることが適当である. 
従って適切に重しをつけた平均の天頂角は, 次式で計算される,
\begin{equation}
  \overline{\cos{\theta_s}}^{day}=\frac{\int_{-h_0}^{h_0} Q\cos{\theta_s}dh}{\int_{-h_0}^{h_0} Qdh }.
\end{equation}
ここで$Q$は(2.14)で与えられる, ある瞬間の日射である.
(2.18)によって計算された一日の平均の太陽天頂角は太陽直下の緯度における最小値$38.3^\circ$とは異なり, 極の暗闇の端では$90^\circ$まで増加する(図 2.8).
極では, 一日の平均の太陽天頂角の最小値は夏至にあたり, そのときは$\phi-\delta=66.55^\circ$である.
高緯度においては平均の太陽天頂角がより大きくなるため, 熱帯の緯度における同じ状況よりも多くの太陽放射が反射される.

\newpage
\markright{2.8 大気上端におけるエネルギバランス}
\section{大気上端におけるエネルギバランス}
地球によって吸収射出されるエネルギーの量は, 地理的そして季節的に異なり, 日射の分布と同様, 大気と地表の状態に依存する.
大気上端におけるエネルギーバランスは単に放射であり, 地球軌道の人工衛星から正確に計測できる.
アルベドは地球のある範囲から反射される太陽放射を計測することと, それと日射とを比較することで見積もられる.
アルベドは興味深い地理的な分布を示す. 
アルベドは, 雲と雪に覆われた部分が豊富であり, 平均した太陽天頂角が大きい極域で最も高い.
アルベドが二番目に最大なのは熱帯と亜熱帯の地域であり, ここでは薄い雲が広く分布している, もしくはサハラ砂漠のように照り返す地表の上である.
アルベドが最も小さいのは, 雲がまばらに分布する熱帯の海の地域である.
海面のアルベドは本来小さい. そのため雲と海氷が存在しないとき, 海のアルベドはたったの$8-10\%$である.
\par
外向きの長波放射(OLR)は, 雲がめったにない温暖な砂漠上と熱帯の海洋上で最大である(図 2.10).
極域と熱帯の曇りが持続的な地域において, 外向きの長波放射は最も小さい.
OLR は射出される物質の温度で決まるので, 冷たい極と冷たい雲の上端は低い値がでる.
温暖な地表が, 比較的乾燥していて雲のない大気に覆われているとき, 最も高い値を取る.
\par
正味の放射量は極の近くで負であり, 熱帯では正である(図 2.11).
最も高い正の値, およそ$120\rm{Wm}^{-2}$は夏半球の亜熱帯の上で生じる.
ここでは多い日射と比較的低いアルベドの両方が太陽放射を多く吸収するのに寄与する.
最大のエネルギーの損失が起こるのは冬半球の極の暗闇であり, ここでは OLR が太陽放射の吸収によって埋め合わせされない.
北アフリカのサハラ砂漠のように乾燥した砂漠の地域は亜熱帯の緯度にも関わらず, 年平均ではエネルギーを失っているので興味深い.
そのエネルギーの欠損は, 乾燥している砂漠の比較的高い地表のアルベドと関係があり, 温暖な地表上の乾いた大気と関連した高い OLR の損失と結びついている.
\par
緯度の一周で平均したとき, 大気上端におけるエネルギーバランスの構成は, 日射の緯度の勾配の効果を明白に示す(図 2.12).
アルベドは緯度に伴って増加するので, 太陽エネルギーの吸収が極に向かって減少するのは, 日射の減少よりもいっそう大きい.
従って, 赤道よりもより高緯度では, より少ない利用できる日射の一部が吸収される.
太陽天頂角や雲の分布, 雪に覆われた地域はすべて緯度に伴って増加するので, アルベドは緯度とともに増加する.
大気によって宇宙空間に射出されるエネルギーは, 太陽放射の吸収と同じくらい急速には緯度とともに減少しない.
それは, 大気と海洋が極に向かって熱を輸送し, 極域における宇宙空間へのエネルギーの損失を補っているからである.
吸収される太陽放射は熱帯での外向きの長波放射(OLR)を上回るので, 熱帯では正味の放射量は正である.
およそ$40^\circ$から極に向かって吸収される太陽放射は OLR を下回り, 正味の放射のバランスは負である.
そのため気候システムは宇宙空間にエネルギーを失う.
年平均した正味の放射量の緯度の勾配は, 地球の気候システムにおいて, 極に向かうエネルギーフラックスによってバランスされなければならない.

\newpage
\markright{2.9 極に向かうエネルギーフラックス}
\section{極に向かうエネルギーフラックス}
図 2.12 の最も低い位置の曲線を見ると, 年平均の正味の放射量が, およそ緯度$40^\circ$から赤道に向かって正であり, 極に向かって負であることが分かる.
図 2.13 に描かれているように, 気候システムに対するエネルギーバランスは, 大気上端における交換, 大気と海洋による該当部分の横の境界への輸送, その部分内でのエネルギーが変化する時間の割合のみが関係している.
固体地球とのエネルギーの交換は無視されうる.
気候システムのエネルギーバランスを,
\begin{equation}
  \DP{E_{ao}}{t}=R_{TOA}-\Delta{F_{ao}}
\end{equation}
のように書くことができる. 
ここで$\partial E_{ao}\/\partial t$は気候システムのエネルギー量の変化率であり, $R_{TOA}$は大気上端における正味の入ってくる放射であり, $\Delta{F_{ao}}$は大気と海洋における水平方向のフラックスの発散である.
\par
一年にわたって平均すると, 貯蔵の項は小さくなり, 大気上端における正味のフラックスと水平の輸送との間での近似的な釣り合いを得る.
\begin{equation}
  R_{TOA}=\Delta{F_{ao}}
\end{equation}
\par
このとき南北方向において必要とされる年平均のエネルギー輸送を求めるために, 図 2.12 の正味の放射量の観測結果を用いることができる. 
正味の放射量を極冠の域まで積分すると, 次のように, 緯度帯ごとの合計のエネルギーフラックスを計算することができる,
\begin{equation}
  \int_{-\frac{\pi}{2}}^\phi \int_0^{2\pi} R_{TOA}a^2\cos{\phi}d{\lambda}d{\phi}=F_{\phi}
\end{equation}
\par
放射の不均衡から推測される北向きの合意のエネルギーフラックスはおよそ$5\times10^{15}\rm{W}$もしくは$5$ペタワット(PW)であり, 中緯度において頂点に達する(図 2.14).
このフラックスは大気と海洋の両方からの寄与を含む.
大気のエネルギーフラックスは, 風, 温度, 湿度についての気球と人工衛星の観測結果から推定され得る. 
このフラックスを合計のフラックスから差し引くと, 海洋のエネルギーフラックスが与えられる. 
海洋の エネルギーフラックスは直接観測されたものから推定することがより難しい.
緯度$30^\circ$では, 大気と海洋の極に向かうフラックスによる寄与は, 各々およそ$2.5\rm{PW}$である.
海洋のフラックスは亜熱帯のおよそ$20^\circ\rm{N}$において頂点に達する一方で, 大気のフラックスは$30^\circ\rm{N}$と$60^\circ\rm{N}$の間に広く最大値をとっている.
6 章と 7 章では大気と海洋がどのようにこの極向きのエネルギーフラックスを成し遂げているかをより詳しく述べる.
地球を包む流体が, 極に向かって熱を輸送しなかったら, 熱帯はより温暖に, 極域はより冷えていただろう.
大気と海洋による熱の輸送は, もしそうでなかったときより, 地球の気候をより均一なものにしている.

\end{document}

