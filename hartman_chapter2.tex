%
%Global Physical Climatology 
%Dennis L. Hartman
%                2016年卒業研究資料 
%                 塩尻千里
%
%
% 2016/10/06 〜 作成
%
% 日付を変更すること
%
%%%%%%%%%%%%%%%%%%%%%%%%%%%%%%%%%%%%%%%%%%%%%%%%%%%%%%%%
%%%%%%%%             Style  Setting             %%%%%%%%
% フォント: 12point (最大), 片面印刷
\documentclass[a4j,12pt,openbib,oneside,dvipdfmx]{jbook}

%%%%%%%%%%%%%%%%%%%%%%%%%%%%%%%%%%%%%%%%%%%%%%%%%%%%%%%%
%%%%%%%%             Package Include            %%%%%%%%
\usepackage{Dennou6}		% 電脳スタイル ver 6
\usepackage{ascmac}
\usepackage{tabularx}
\usepackage{graphicx}
\usepackage{amssymb}
\usepackage{amsmath}
\usepackage{bm}

%%%%%%%%%%%%%%%%%%%%%%%%%%%%%%%%%%%%%%%%%%%%%%%%%%%%%%%%
%%%%%%%%            PageStyle Setting           %%%%%%%%
\pagestyle{DAmyheadings}


%%%%%%%%%%%%%%%%%%%%%%%%%%%%%%%%%%%%%%%%%%%%%%%%%%%%%%%%
%%%%%%%%        Title and Auther Setting        %%%%%%%%
%%
%%  [ ] はヘッダに書き出される.
%%  { } は表題 (\maketitle) に書き出される.

\Dtitle{Hartman(1994)}   % 変更不可
\Dauthor{塩尻千里}       % ゼミ担当者の名前
\Ddate{2016/10/06~}      % ゼミの日時 (毎回変更すること)
\Dfile{hartman\_chapter2.tex}  %file名




%%%%%%%%%%%%%%%%%%%%%%%%%%%%%%%%%%%%%%%%%%%%%%%%%%%%%%%%
%%%%%%%%        Counter Output Format           %%%%%%%%
\def\thechapter{\arabic{chapter}}
\def\thesection{\arabic{chapter}.\arabic{section}}
\def\thesubsection{\arabic{chapter}.\arabic{section}.\arabic{subsection}}
\def\theequation{\arabic{chapter}.\arabic{equation}}
\def\thepage{\arabic{page}}
\def\thefigure{\arabic{chapter}.\arabic{figure}}
\def\thetable{\arabic{chapter}.\arabic{section}.\arabic{table}}
\def\thefootnote{*\arabic{footnote}}

%%%%%%%%%%%%%%%%%%%%%%%%%%%%%%%%%%%%%%%%%%%%%%%%%%%%%%%%
%%%%%%%%   Set Counter (chapter, section etc. ) %%%%%%%%
\setcounter{chapter}{1}    % 章番号
\setcounter{section}{0}    % 節番号
\setcounter{subsection}{0}    % 節番号
\setcounter{equation}{0}   % 式番号
\setcounter{page}{1}     % 必ず開始ページは明記する
\setcounter{figure}{0}     % 図番号
\setcounter{table}{0}      % 表番号
\setcounter{footnote}{0}

%%%%%%%%%%%%%%%%%%%%%%%%%%%%%%%%%%%%%%%%%%%%%%%%%%%%%%%%
%%%%%%%%        Dennou-Style Definition         %%%%%%%%

%% 改段落時の空行設定
\Dparskip      % 改段落時に一行空行を入れる
%\Dnoparskip    % 改段落時に一行空行を入れない

%% 改段落時のインデント設定
\Dparindent    % 改段落時にインデントする
%\Dnoparindent  % 改段落時にインデントしない

%%%%%%%%%%%%%%%%%%%%%%%%%%%%%%%%%%%%%%%%%%%%%%%%%%%%%%%%
%%%%%%%%             Text Start                 %%%%%%%%
\begin{document}
\markright{2.1 熱とエネルギー}

\chapter{全体のエネルギーバランス}	
\section{熱とエネルギー}
\par
気候変動の鍵である温度は, 分子運動に含まれるエネルギーの単位である. どのように温度が維持されるかを理解するためには, 熱力学第一法則で正式に述べられているエネルギーバランスを考えなければならない. 
基本的な地球の全体のエネルギーバランスは, 太陽からくるエネルギーと, 地球の放射によって宇宙空間に戻されるエネルギー間のつり合いである. 地球内部におけるエネルギーの生成は, そのエネルギー収支においては無視できるほどの影響である. 
太陽放射の吸収はほとんど地表面で行われるが, 一方で, 宇宙空間への放射のほとんどは大気から生じる. 地球大気は効率的に赤外線放射を吸収, 放射するので, 地表面は大気が存在しなかった場合に比べてより温暖である. 
一年を通じて平均すると, 太陽エネルギーは, 極付近よりも赤道付近でより多く吸収されている. 大気と海洋は, 表面温度に対する熱の勾配の効果を減らすように極へエネルギーを輸送する. 地球の進化や気候の特徴の大半は, 太陽系内における地球の位置によって決定されてきた.

\newpage
\section{太陽系}
\markright{2.2 太陽系}
\par
地球上で生命を維持するためのエネルギー源は太陽からくる. 地球は太陽からの距離を比較的一定に保ちながら一年をかけて太陽の周囲を軌道を描いて回る. そのため私たちの太陽は安定で快適な熱と光の資源を提供する. 私たちの太陽は, 私たちの銀河である天の川において, 約$10^{11}$個ある星のうちの一つである. それは一つの星であるが, 一方で, 目に見える星の$2/3$は多重星の系にある.
\par
光度とは, 太陽によってエネルギーが放射されたところにおける合計の比率である. 
太陽に比べて$10^{-4}$倍の光度がある恒星や, $10^5$倍の輝度をもつ恒星があることを私たちは知っている. それらの温度は 2000 ~ 30000 K にわたって分布している一方で, 太陽の光球の温度はおよそ 6000 K である. 
光球は, 太陽のエネルギー放射の大部分を宇宙空間に放射する領域である. 恒星の半径は, 0.1 から 200 太陽半径に及ぶ. エネルギーは核融合によって太陽の核で生成され, そこでより軽い元素が重い元素にされ, その過程でエネルギーの放出が伴う.
太陽のようなやや小さい星の場合, 主な順序において予想される寿命はおよそ 110 億年であり, およそ半分が過ぎてしまった. 従って太陽はさびしく, 中年で, 中くらいの輝きの星である. 
星の進化の理論は, 太陽の光度は地球の一生, およそ 50 億年の間, 約 30$\%$で増加してきたと予測している. 
\par
太陽系は 9 つの惑星を含む. それらは地球型もしくは内側の惑星と, ヨブもしくは外側の惑星(図2.2)とに分類されうる. 地球型の惑星には, 水星, 金星, 火星, 地球が含まれる. ヨブの惑星には木星, 土星, 天王星, 海王星が含まれる. 冥王星はどちらのカテゴリーのもきちんと分類されない. \\
\subsection{惑星の運動}
\par
惑星は太陽の周りを楕円軌道を描いて回る. そしてそれには三つの特徴がある: 惑星太陽間の距離の平均, 離心率, 軌道の水平面の方向である. 太陽からの距離の平均は, 惑星に到達する太陽のエネルギーフラックス密度(単位時間あたり単位面積あたりに届くエネルギー)の量を決める. 太陽からの距離の平均は, また, 惑星の一年つまり惑星が一回軌道を回り切るのにかかる時間を決める. 惑星の一年は太陽からの距離が大きくなるにつれて増える.
\par
軌道の離心率とは, 完全な円軌道からどれだけ軌道がそれているかの程度のことである. 惑星が惑星の一年間にその軌道を動くのに伴う, 惑星に到達する太陽のエネルギーフラックス密度の変化量を決める. 
もし離心率がゼロ(円軌道)でないならば, 太陽からの惑星の距離は一年を通じて変化する. 軌道の水平方向の向きは気候に対してそこまで支配的な影響がない. 冥王星を除いた多くの惑星は, 程度の差はあれど同じ軌道面にいる.
\par
軌道のパラメタに加えて, 惑星の自転のパラメタと軌道との関係はとても重要である. 自転率は, ある地点における日射の時間的な振る舞い(毎日のサイクル)を決める. 
そしてまた, 太陽の熱に対する大気と海洋の応答を決める上で重要であり, それによって風と発達する流れのパターンを決める.
\par
傾斜もしくは傾きは, 自転軸と, 軌道の水平面の法線方向間の角度である. それはとりわけ高緯度において, 日射の季節変化に影響を与える. それはまた極域に届く日射の年間平均に強く影響する. 現在地球の自転軸の傾きはおよそ$23.45^\circ$である.
\par
近日点の経度は, 軌道上における惑星の位置に関連した季節の段階を示す. たとえば現在は, 南半球が夏の間, およそ 1 月 5 日に地球は太陽の最も近くを通る(近日点). 結果として南半球は北半球に比べてより多く, 大気上層の日射を受け取る.
\par
気象におけるこれらの軌道パラメタの効果は, 11 章でより詳しく議論されるだろう. そこでは気候変動に関する軌道パラメタの理論が記述されている. ここでは太陽からの距離と, 傾斜もしくは偏角だけを考える.

\newpage
\section{地球のエネルギーバランス}
\markright{2.3 地球のエネルギーバランス}
\subsection{熱力学第一法則}
\par
熱力学第一法則はエネルギーが保存することを述べている. 閉じた系における第一法則は, 「ある系に加えられた熱は内部エネルギーの変化引く出された仕事に等しい」と記述されるかもしれない. この法則は, 記号では
\begin{equation}
  dQ=dU-dW
\end{equation}
と記述されうる.
\par
ここで$dQ$は加えられた熱量, $dU$は系の内部エネルギーの変化, $dW$は系から出て行った仕事である.
\par
熱は三つの方法で系に輸送され, 輸出されうる.
\begin{enumerate}
  \item{\textgt{放射}: 質量が変化せず, 媒質は必要とされない. 純粋なエネルギーは光速で動く.}
  \item{\textgt{伝導}: 質量は変化しないが, 原子もしくは分子同士の衝突によって熱を輸送するため. 媒質が必要とされる.}
  \item{\textgt{対流}: 質量が変化する. 網目状に質量の運動が起こる. しかしより一般には, 異なるエネルギー量をもつパーセルが場所を変えるので, 網目状の質量の運動は起こらず, エネルギーが変化する(図2.1).}
\end{enumerate}
\par
太陽から地球へのエネルギーの伝達はほとんどすべて放射である. 質量フラックスは太陽風の粒子と関連するものもあるが, エネルギー量は地球の表面温度に対して重要な効果をもつには小さすぎる. 
さらに, 地球環境において地球によってなされた仕事もまた, 無視できるほどである. 地球のエネルギーバランスを近似的に計算するためには, 放射エネルギーのみを考える必要がある. 
地球と太陽の間のエネルギーの流れに影響を与えうるような宇宙空間の物質の量は小さい. 従って, 太陽の光球と地球大気の上端との間の宇宙空間を真空と考える. 
真空では放射だけがエネルギーを輸送しうる. 

\subsection{エネルギーフラックス, フラックス密度, 太陽定数}
\par
太陽はほぼ一定でエネルギーフラックスを放出しており, それを私たちは光度とよび, $L_0=3.9\times10^{26}$ Wである.
エネルギーフラックスを光球の面積で割ることで, 光球におけるフラックス密度の平均を計算できる. 
\begin{align}
  \rm{Flux\, density_{photo}}&=\frac{\rm{flux}}{\rm{area_{photo}}}=\frac{L_0}{4\pi{r^2_{\rm{photo}}}} \nonumber \\
  &=\frac{3.9\times10^{26}\,\rm{W}}{4\pi\left[6.96\times10^8\,\rm{m}\right]^2}=6.4\times10^7\,\rm{Wm^{-2}}
\end{align}
\par
宇宙空間はだいたい真空であり, エネルギーは保存されるので, \\
は光度もしくは太陽からのエネルギーフラックスの合計に等しい. フラックス密度が球上で一様だと仮定し, 太陽からある距離$d$におけるフラックス密度を$S_d$と書くと, エネルギー保存より,
\begin{equation}
  {\rm{Flux}}={L_0}={S_d}4\pi{d^2}
\end{equation}
である.
\par
これより, フラックス密度は太陽までの距離の平方根の逆数に比例することが分かる. いま太陽定数をある特定の距離における太陽放射のエネルギーフラックス密度と定義する.
\begin{equation}
  {\rm{Solar}}\,{\rm{constant}}={\rm{flux}}\,{\rm{density}}\,{\rm{at}}\,{\rm{distanse}}\, d=S_d=\frac{L_0}{4\pi{d^2}}
\end{equation}
太陽定数は, ある固定された半径の球面において, また the sun at its center でのみ一定である. 太陽から地球までの平均の距離($1.5\times10^{11}\,\rm{m}$)においては, 太陽定数は$S_0=1367\,\rm{Wm}^{-2}$である.

\subsection{空洞の放射}
\par
熱力学的に平衡な状態にある, ある閉じた空洞内の放射領域は, 空洞の壁の温度それぞれに相関のある値をもち, 空洞がつくられている物質には関係しない. 
この空洞の放射強度, それは壁の温度それぞれに相関があるのだが, それはまた黒体放射と呼ばれる. なぜなら, 表面からの放射に単位射出量が一致するからである. 
完全な黒体は容易に見つからないかもしれないが, 平衡状態にある空洞内部の放射は常に黒体放射と等しいだろう. 
黒体放射の温度の依存性はステファンボルツマンの法則に従う.
\begin{equation}
  {\rm{E_{BB}}}=\sigma{T^4}\,\,;\,\,\sigma=5.67\times10^{-8}\,\rm{Wm}^{-2}\rm{K}^{-4}
\end{equation}

\subsubsection{例: 太陽の放射温度}
光球における太陽のフラックス密度はおよそ$6.4\times10^7\,\rm{Wm}^{-2}$と私たちは容易に計算した. 私たちはこれをステファンボルツマンの式と等しくすることができ, 光球についての有効放射温度を導くことができる. 

\subsection{惑星の放射温度}


\end{document}

